%!TEX root = ../dokumentation.tex

\pagestyle{empty}

\renewcommand{\abstractname}{\langabstract} % Text für Überschrift

%Zum Abstract
%Essenz / komprimierte Wiedergabe des Inhaltes
%Kernaussagen, Folgerungen, keine Bewertung
%Eine Seite (200-250 Wörter); zum Schluss schreiben

% \begin{otherlanguage}{english} % auskommentieren, wenn Abstract auf Deutsch sein soll
\begin{abstract}
Von Web Tracking spricht man, sobald die Verfolgung durch Drittparteien mit dem Ziel Informationen über den Besucher zu erfassen stattfindet \cite[vgl. ScSE14]{fraunhofer:2014}.
Der Unterschied ist für den durchschnittlichen Nutzer jedoch nicht ersichtlich, da technische Kenntnisse erforderlich sind, um zu erfassen mit wem der Browser eine Verbindung aufbaut und welche Daten darüber gesendet werden. Zudem verschwimmen die Grenzen zwischen Web Analyse und Web Tracking, denn zur Analyse werden oft Produkte von Drittanbietern (etwa Google Analytics \cite{google-analytics:2018}) verwendet, welche die Daten aufbereiten und bereitstellen. Dieser Prozess ist für den Nutzer nicht transparent.\newline
Zu den negativen Auswirkungen auf die Gesellschaft zählt die massive Verletzung der bürgerlichen Privatsphäre, denn die Methoden der Web Analyse und des Web Trackings erfolgen in den meisten Fällen ohne die explizite Erlaubnis der Nutzer. Die vorgestellten Praktiken bieten Potential zur Manipulation der Gesellschaft, sei es durch nutzerspezifische, maßgeschneiderte Werbung die zu erhöhtem Konsumverhalten führt, oder politischer Manipulation über soziale Netzwerke.\newline
Das Prinzip, welches sozialen Netzwerken zugrunde liegt, nämlich die Tatsache, dass Nutzern ähnliche Inhalte präsentiert werden, ist besonders bei politischen und extremistischen Inhalten gefährlich. Nutzer, die bereits extremistische Standpunkte vertreten, sehen sich bestätigt, da sie ähnliche Inhalte von anderen Nutzern sehen.\newline
Nach Aussage von Mark Zuckerberg, dem CEO von Facebook \cite{facebook}, wurden die Nutzer des Dienstes im Zeitraum der US-Amerikanischen Präsidentschaftswahl 2016 durch die russische Agentur IRA (Internet Research Agency) gezielt getäuscht um die Wahl zu beeinflussen \cite[vgl. Zuck18]{zuck:2018}.\newline
Eine Möglichkeit die negativen Effekte von Web Analysen abzuschwächen sind staatliche Regulieren, wie sie bereits in der neuen EU Grundschutzverordnung festgehalten sind \cite[vgl. EU16]{eu-datenschutz-grundverordnung}, welche zum 25. Mai 2018 in Kraft treten wird. Eine weitere Möglichkeit ist das Unterstützen nichtstaatlicher, gemeinnützigen Organisationen, deren Ziel es ist die Privatsphäre der Bürger zu schützen. Ein Beispiel hierfür ist die Electronic Frontier Foundation (EFF) \cite{eff}.

%Struktur:
%<Thema, Nennung des Titels>
%<State of the Art, Ziel, Fragestellung>
%<Vorgehensweise, Methodik>
%<Ergebnisse werden evtl. vorausgenommen> oder
%<Hinweise auf Art der Ergebnisse geben>
\end{abstract}
