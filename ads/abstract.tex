%!TEX root = ../dokumentation.tex

\pagestyle{empty}

\renewcommand{\abstractname}{\langabstract} % Text für Überschrift

%Zum Abstract
%Essenz / komprimierte Wiedergabe des Inhaltes
%Kernaussagen, Folgerungen, keine Bewertung
%Eine Seite (200-250 Wörter); zum Schluss schreiben

% \begin{otherlanguage}{english} % auskommentieren, wenn Abstract auf Deutsch sein soll
\begin{abstract}
\textbf{Sascha Hug}:\newline
In der heutigen Zeit ist das Internet nicht mehr wegzudenken, es gibt für alles eine Webseite im Internet. Jedoch sollen auch diese Webseiten optimiert werden, weshalb es sogenannte Webanalyse gibt. Der aktuelle Facebook-Daten-Skandal wurde auf Basis einer Datenanalysefirma namens Cambridge Analytica verursacht, hier wurden Daten aus 50 Millionen Facebook-Profilen illegal genutzt, wohl u.a. auch zur vermeintlichen Manipulation der US-Präsidentschaftswahl und des Brexit-Votums \cite{face:2018}. Anhand dieses aktuellen Beispiels wird deutlich, dass die Auswirkung auf die Gesellschaft üble Folgen mit sich ziehen können. Eine Webanalyse auch Webcontrolling genannt, dient zu einer Effizienzsteigerung der Webseite. Bei einer Webanalyse wird das Verhalten der Benutzer im Web erfasst und ausgewertet, diese Informationen werden in sogenannten Kennzahlen gespeichert und analysiert. Diese Leistungskennzahlen wie Besucheranzahl, Verweildauer oder Page Value können verwendet werden, um auf Online-Trends zu reagieren. Resümiert lässt sich die Webanalyse auf jegliche Geschäftsmodelle anpassen, um eine Gewinnmaximierung zu erreichen. Als Ergebnis der Webanalyse erfolgt eine Klassifizierung von Besuchern, dies bedeutet eine Strukturierung oder Kategorisierung der Benutzer. Benutzer können zusammengefasst und als Produkt abgestempelt werden. Die Rechtslage bei der Nutzung von Analytic-Tools ist aktuell noch sehr umstritten. Rechtlicher Impuls dafür ist meist das Speichern der IP-Adresse und das Verwenden von Cookies. Das Bundesdatenschutzgesetz lässt nur dann die Speicherung von personenbezogenen Daten zu, wenn dies von einer gesetzlichen Vorschrift explizit erlaubt wird oder eine Einwilligung des Nutzers vorliegt. Relevant ist dabei die Regelung in § 15 Telemediengesetz (TMG).  Demnach dürfen personenbezogene Daten von Besuchern einer Internetseite ohne deren Einwilligung nur unter Verwendung eines Pseudonym erhoben werden. Nach § 13 Telemediengesetz (TMG) haben Anbieter von Internetportalen zu gewährleisten, dass „die anfallenden personenbezogenen Daten über den Ablauf des Zugriffs oder der sonstigen Nutzung unmittelbar nach deren Beendigung gelöscht“ werden. Aktuell liegt dieses Thema noch beim Europäischen Gerichtshof in Luxemburg, dort soll eine Lösung für die “Grauzone” gefunden werden. Jedoch stellt sich dann in Zukunft die Frage, ob dieses Urteil Auswirkungen auf unsere tägliche Internetnutzung hat \cite{ryte:2018}.

\textbf{Tobias Lamm}: \newline
Von Web Tracking spricht man, sobald die Verfolgung durch Drittparteien mit dem Ziel Informationen über den Besucher zu erfassen stattfindet \cite[vgl. ScSE14]{fraunhofer:2014}.
Der Unterschied ist für den durchschnittlichen Nutzer jedoch nicht ersichtlich, da technische Kenntnisse erforderlich sind, um zu erfassen mit wem der Browser eine Verbindung aufbaut und welche Daten darüber gesendet werden. Zudem verschwimmen die Grenzen zwischen Web Analyse und Web Tracking, denn zur Analyse werden oft Produkte von Drittanbietern (etwa Google Analytics \cite{google-analytics:2018}) verwendet, welche die Daten aufbereiten und bereitstellen. Dieser Prozess ist für den Nutzer nicht transparent.

\textbf{Olga Akymenko}:\newline
Die positiven Auswirkungen auf die Gesellschaft lassen sich im wesentlichen in zwei Kategorien einteilen:
\begin{enumerate}
	\item Verbesserung der Kundenzufriedenheit
	\begin{enumerate}
		\item durch Usability
		\item durch zum Nutzer passender Content 
	\end{enumerate}
	\item Strategien zur Optimierung Webseite
		\begin{enumerate}
		\item Monitoring der Effektivität
		\item Steigerung der Effektivität 
	\end{enumerate}
\end{enumerate}

Im Bezug auf Kundenzufriedenheit spielt Usability eine wichtige Rolle. Web Analyse ermöglicht es, die Wünsche von Nutzern zu verstehen und die Struktur der Seite klarer zu machen (z.B welche Farbe soll der Button “Kaufen“ haben und wo soll die Suchleiste sein, damit der Nutzer sie einfacher findet). 
Ein weiterer Punkt sind Inhalte. In der gegenwärtigen Situation der Medien erhalten Konsumenten zu viele Information auf einmal. Medienkonsumenten sind mit den Inhalten aus verschiedenen Kanälen überfordert. In diesem Fall können gezielte Werbung und passender Content als positive Auswirkungen gesehen werden, da sie dem Nutzer ein angehmeres Surferlebnis bieten.

Weiterhin werden durch Web Analysen Strategien zur Optimierung der Webseite ermöglicht. Um eine erfolgreiche Webseite zu gestalten, können Betreiber auf bestimmte Kennzahlen der Web Analyse stützen: besonders beliebte oder unbeliebte Seiten einer Website geben einen guten Richtwert dafür, was der Besucher sehen möchte. Daraus kann eine Optimierung nach Inhalt und Gestaltung der Seite erfolgen. Zudem können Besuchergruppen voneinander unterschieden werden. Damit können tiefere Einblicke ins Nutzerverhalten gewonnen werden, vor allen in die Klick- und Prozessanalyse. 
Die Messung und Optimierung von Seitenabfolgen einer Website \cite[Ecomm18]{ecommerce} bietet gute Kennzahlen, wenn es um die Erhörung und das Monitoring der Effektivität der Webseite geht. So lassen sich Kennziffern wie Kosten, Umsatz, Konvensionsrate, Pageviews pro Sitzung, Sitzungen pro Besucher zu sinnvollen Informationen verdichten und in einen Kontext der Vergleichbarkeit setzen: Der Verlauf des Umsatzes über das Jahr, Kosten pro Kampagne, Konversionsrate im Vergleich zu einem gesetzten Ziel etc.

\textbf{Tobias Lamm}:\newline
Zu den negativen Auswirkungen auf die Gesellschaft zählt die massive Verletzung der bürgerlichen Privatsphäre, denn die Methoden der Web Analyse und des Web Trackings erfolgen in den meisten Fällen ohne die explizite Erlaubnis der Nutzer. Die vorgestellten Praktiken bieten Potential zur Manipulation der Gesellschaft, sei es durch nutzerspezifische, maßgeschneiderte Werbung die zu erhöhtem Konsumverhalten führt, oder politischer Manipulation über soziale Netzwerke.\newline
Das Prinzip, welches sozialen Netzwerken zugrunde liegt, nämlich die Tatsache, dass Nutzern ähnliche Inhalte präsentiert werden, ist besonders bei politischen und extremistischen Inhalten gefährlich. Nutzer, die bereits extremistische Standpunkte vertreten, sehen sich bestätigt, da sie ähnliche Inhalte von anderen Nutzern sehen.\newline
Nach Aussage von Mark Zuckerberg, dem CEO von Facebook \cite{facebook}, wurden die Nutzer des Dienstes im Zeitraum der US-Amerikanischen Präsidentschaftswahl 2016 durch die russische Agentur IRA (Internet Research Agency) gezielt getäuscht um die Wahl zu beeinflussen \cite[vgl. Zuck18]{zuck:2018}.\newline
Eine Möglichkeit die negativen Effekte von Web Analysen abzuschwächen sind staatliche Regulieren, wie sie bereits in der neuen EU Grundschutzverordnung festgehalten sind \cite[vgl. EU16]{eu-datenschutz-grundverordnung}, welche zum 25. Mai 2018 in Kraft treten wird. Eine weitere Möglichkeit ist das Unterstützen nichtstaatlicher, gemeinnützigen Organisationen, deren Ziel es ist die Privatsphäre der Bürger zu schützen. Ein Beispiel hierfür ist die Electronic Frontier Foundation (EFF) \cite{eff}.


\textbf{Olga Akymenko}: \newline
Nach dem einige der positiven und negativen Auswirkungen betrachten wurden, lässt sich sagen, dass Nutzer auf der einen Seite Bequemlichkeit und Aufmerksamkeit bekommt. Auf der anderen wird er in seiner Privatssphäre eingeschränkt, denn und der Nutzer weiß nicht genau, welche Daten und vor allem wofür benutzt werden. Allerdings lassen sich auch positiv Szenarien aufzeigen, beispielsweise kann die Web Analyse von Bildungseinrichtungen genutzt werden um zu erkennen, auf welche Inhalte sich der Betreiber fokussieren sollte. Auch für staatliche, entgeltfreie Dienste, in denen die Daten der Web Analyse nur für gesellschaftlichen Nutzen erhoben werden und nicht für Kommerzielle Zwecke, bieten Potential für diese Technologie.
%Struktur:
%<Thema, Nennung des Titels>
%<State of the Art, Ziel, Fragestellung>
%<Vorgehensweise, Methodik>
%<Ergebnisse werden evtl. vorausgenommen> oder
%<Hinweise auf Art der Ergebnisse geben>
\end{abstract}
